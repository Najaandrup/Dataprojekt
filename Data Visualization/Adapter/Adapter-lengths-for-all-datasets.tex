% Options for packages loaded elsewhere
\PassOptionsToPackage{unicode}{hyperref}
\PassOptionsToPackage{hyphens}{url}
%
\documentclass[
]{article}
\usepackage{amsmath,amssymb}
\usepackage{iftex}
\ifPDFTeX
  \usepackage[T1]{fontenc}
  \usepackage[utf8]{inputenc}
  \usepackage{textcomp} % provide euro and other symbols
\else % if luatex or xetex
  \usepackage{unicode-math} % this also loads fontspec
  \defaultfontfeatures{Scale=MatchLowercase}
  \defaultfontfeatures[\rmfamily]{Ligatures=TeX,Scale=1}
\fi
\usepackage{lmodern}
\ifPDFTeX\else
  % xetex/luatex font selection
\fi
% Use upquote if available, for straight quotes in verbatim environments
\IfFileExists{upquote.sty}{\usepackage{upquote}}{}
\IfFileExists{microtype.sty}{% use microtype if available
  \usepackage[]{microtype}
  \UseMicrotypeSet[protrusion]{basicmath} % disable protrusion for tt fonts
}{}
\makeatletter
\@ifundefined{KOMAClassName}{% if non-KOMA class
  \IfFileExists{parskip.sty}{%
    \usepackage{parskip}
  }{% else
    \setlength{\parindent}{0pt}
    \setlength{\parskip}{6pt plus 2pt minus 1pt}}
}{% if KOMA class
  \KOMAoptions{parskip=half}}
\makeatother
\usepackage{xcolor}
\usepackage[margin=1in]{geometry}
\usepackage{color}
\usepackage{fancyvrb}
\newcommand{\VerbBar}{|}
\newcommand{\VERB}{\Verb[commandchars=\\\{\}]}
\DefineVerbatimEnvironment{Highlighting}{Verbatim}{commandchars=\\\{\}}
% Add ',fontsize=\small' for more characters per line
\usepackage{framed}
\definecolor{shadecolor}{RGB}{248,248,248}
\newenvironment{Shaded}{\begin{snugshade}}{\end{snugshade}}
\newcommand{\AlertTok}[1]{\textcolor[rgb]{0.94,0.16,0.16}{#1}}
\newcommand{\AnnotationTok}[1]{\textcolor[rgb]{0.56,0.35,0.01}{\textbf{\textit{#1}}}}
\newcommand{\AttributeTok}[1]{\textcolor[rgb]{0.13,0.29,0.53}{#1}}
\newcommand{\BaseNTok}[1]{\textcolor[rgb]{0.00,0.00,0.81}{#1}}
\newcommand{\BuiltInTok}[1]{#1}
\newcommand{\CharTok}[1]{\textcolor[rgb]{0.31,0.60,0.02}{#1}}
\newcommand{\CommentTok}[1]{\textcolor[rgb]{0.56,0.35,0.01}{\textit{#1}}}
\newcommand{\CommentVarTok}[1]{\textcolor[rgb]{0.56,0.35,0.01}{\textbf{\textit{#1}}}}
\newcommand{\ConstantTok}[1]{\textcolor[rgb]{0.56,0.35,0.01}{#1}}
\newcommand{\ControlFlowTok}[1]{\textcolor[rgb]{0.13,0.29,0.53}{\textbf{#1}}}
\newcommand{\DataTypeTok}[1]{\textcolor[rgb]{0.13,0.29,0.53}{#1}}
\newcommand{\DecValTok}[1]{\textcolor[rgb]{0.00,0.00,0.81}{#1}}
\newcommand{\DocumentationTok}[1]{\textcolor[rgb]{0.56,0.35,0.01}{\textbf{\textit{#1}}}}
\newcommand{\ErrorTok}[1]{\textcolor[rgb]{0.64,0.00,0.00}{\textbf{#1}}}
\newcommand{\ExtensionTok}[1]{#1}
\newcommand{\FloatTok}[1]{\textcolor[rgb]{0.00,0.00,0.81}{#1}}
\newcommand{\FunctionTok}[1]{\textcolor[rgb]{0.13,0.29,0.53}{\textbf{#1}}}
\newcommand{\ImportTok}[1]{#1}
\newcommand{\InformationTok}[1]{\textcolor[rgb]{0.56,0.35,0.01}{\textbf{\textit{#1}}}}
\newcommand{\KeywordTok}[1]{\textcolor[rgb]{0.13,0.29,0.53}{\textbf{#1}}}
\newcommand{\NormalTok}[1]{#1}
\newcommand{\OperatorTok}[1]{\textcolor[rgb]{0.81,0.36,0.00}{\textbf{#1}}}
\newcommand{\OtherTok}[1]{\textcolor[rgb]{0.56,0.35,0.01}{#1}}
\newcommand{\PreprocessorTok}[1]{\textcolor[rgb]{0.56,0.35,0.01}{\textit{#1}}}
\newcommand{\RegionMarkerTok}[1]{#1}
\newcommand{\SpecialCharTok}[1]{\textcolor[rgb]{0.81,0.36,0.00}{\textbf{#1}}}
\newcommand{\SpecialStringTok}[1]{\textcolor[rgb]{0.31,0.60,0.02}{#1}}
\newcommand{\StringTok}[1]{\textcolor[rgb]{0.31,0.60,0.02}{#1}}
\newcommand{\VariableTok}[1]{\textcolor[rgb]{0.00,0.00,0.00}{#1}}
\newcommand{\VerbatimStringTok}[1]{\textcolor[rgb]{0.31,0.60,0.02}{#1}}
\newcommand{\WarningTok}[1]{\textcolor[rgb]{0.56,0.35,0.01}{\textbf{\textit{#1}}}}
\usepackage{graphicx}
\makeatletter
\def\maxwidth{\ifdim\Gin@nat@width>\linewidth\linewidth\else\Gin@nat@width\fi}
\def\maxheight{\ifdim\Gin@nat@height>\textheight\textheight\else\Gin@nat@height\fi}
\makeatother
% Scale images if necessary, so that they will not overflow the page
% margins by default, and it is still possible to overwrite the defaults
% using explicit options in \includegraphics[width, height, ...]{}
\setkeys{Gin}{width=\maxwidth,height=\maxheight,keepaspectratio}
% Set default figure placement to htbp
\makeatletter
\def\fps@figure{htbp}
\makeatother
\setlength{\emergencystretch}{3em} % prevent overfull lines
\providecommand{\tightlist}{%
  \setlength{\itemsep}{0pt}\setlength{\parskip}{0pt}}
\setcounter{secnumdepth}{-\maxdimen} % remove section numbering
\ifLuaTeX
  \usepackage{selnolig}  % disable illegal ligatures
\fi
\usepackage{bookmark}
\IfFileExists{xurl.sty}{\usepackage{xurl}}{} % add URL line breaks if available
\urlstyle{same}
\hypersetup{
  pdftitle={Adapter lengths for all datasets},
  hidelinks,
  pdfcreator={LaTeX via pandoc}}

\title{Adapter lengths for all datasets}
\author{}
\date{\vspace{-2.5em}2025-03-19}

\begin{document}
\maketitle

\subsubsection{Load all data}\label{load-all-data}

\begin{Shaded}
\begin{Highlighting}[]
\NormalTok{a60\_unmod }\OtherTok{\textless{}{-}} \FunctionTok{read.table}\NormalTok{(}\AttributeTok{file =} \StringTok{"a60\_unmod\_polyA\_position.tsv"}\NormalTok{, }\AttributeTok{sep =} \StringTok{"}\SpecialCharTok{\textbackslash{}t}\StringTok{"}\NormalTok{, }\AttributeTok{header =} \ConstantTok{TRUE}\NormalTok{)}
\NormalTok{a60\_30 }\OtherTok{\textless{}{-}} \FunctionTok{read.table}\NormalTok{(}\AttributeTok{file =} \StringTok{"a60\_30\_polyA\_position.tsv"}\NormalTok{, }\AttributeTok{sep =} \StringTok{"}\SpecialCharTok{\textbackslash{}t}\StringTok{"}\NormalTok{, }\AttributeTok{header =} \ConstantTok{TRUE}\NormalTok{)}
\NormalTok{a60\_60 }\OtherTok{\textless{}{-}} \FunctionTok{read.table}\NormalTok{(}\AttributeTok{file =} \StringTok{"a60\_60\_polyA\_position.tsv"}\NormalTok{, }\AttributeTok{sep =} \StringTok{"}\SpecialCharTok{\textbackslash{}t}\StringTok{"}\NormalTok{, }\AttributeTok{header =} \ConstantTok{TRUE}\NormalTok{)}

\NormalTok{a120\_unmod }\OtherTok{\textless{}{-}} \FunctionTok{read.table}\NormalTok{(}\AttributeTok{file =} \StringTok{"a120\_unmod\_polyA\_position.tsv"}\NormalTok{, }\AttributeTok{sep =} \StringTok{"}\SpecialCharTok{\textbackslash{}t}\StringTok{"}\NormalTok{, }\AttributeTok{header =} \ConstantTok{TRUE}\NormalTok{)}
\NormalTok{a120\_1mod }\OtherTok{\textless{}{-}} \FunctionTok{read.table}\NormalTok{(}\AttributeTok{file =} \StringTok{"a120\_1mod\_polyA\_position.tsv"}\NormalTok{, }\AttributeTok{sep =} \StringTok{"}\SpecialCharTok{\textbackslash{}t}\StringTok{"}\NormalTok{, }\AttributeTok{header =} \ConstantTok{TRUE}\NormalTok{)}
\NormalTok{a120\_2mod }\OtherTok{\textless{}{-}} \FunctionTok{read.table}\NormalTok{(}\AttributeTok{file =} \StringTok{"a120\_2mod\_polyA\_position.tsv"}\NormalTok{, }\AttributeTok{sep =} \StringTok{"}\SpecialCharTok{\textbackslash{}t}\StringTok{"}\NormalTok{, }\AttributeTok{header =} \ConstantTok{TRUE}\NormalTok{)}
\NormalTok{a120\_4mod }\OtherTok{\textless{}{-}} \FunctionTok{read.table}\NormalTok{(}\AttributeTok{file =} \StringTok{"a120\_4mod\_polyA\_position.tsv"}\NormalTok{, }\AttributeTok{sep =} \StringTok{"}\SpecialCharTok{\textbackslash{}t}\StringTok{"}\NormalTok{, }\AttributeTok{header =} \ConstantTok{TRUE}\NormalTok{)}
\end{Highlighting}
\end{Shaded}

\subsubsection{Define lengths of
adapter}\label{define-lengths-of-adapter}

\begin{Shaded}
\begin{Highlighting}[]
\NormalTok{a60\_unmod[}\StringTok{"adapter\_length"}\NormalTok{] }\OtherTok{\textless{}{-}}\NormalTok{ a60\_unmod}\SpecialCharTok{$}\NormalTok{start }\SpecialCharTok{{-}} \DecValTok{1}
\NormalTok{a60\_30[}\StringTok{"adapter\_length"}\NormalTok{] }\OtherTok{\textless{}{-}}\NormalTok{ a60\_30}\SpecialCharTok{$}\NormalTok{start }\SpecialCharTok{{-}} \DecValTok{1}
\NormalTok{a60\_60[}\StringTok{"adapter\_length"}\NormalTok{] }\OtherTok{\textless{}{-}}\NormalTok{ a60\_60}\SpecialCharTok{$}\NormalTok{start }\SpecialCharTok{{-}} \DecValTok{1}

\NormalTok{a120\_unmod[}\StringTok{"adapter\_length"}\NormalTok{] }\OtherTok{\textless{}{-}}\NormalTok{ a120\_unmod}\SpecialCharTok{$}\NormalTok{start }\SpecialCharTok{{-}} \DecValTok{1}
\NormalTok{a120\_1mod[}\StringTok{"adapter\_length"}\NormalTok{] }\OtherTok{\textless{}{-}}\NormalTok{ a120\_1mod}\SpecialCharTok{$}\NormalTok{start }\SpecialCharTok{{-}} \DecValTok{1}
\NormalTok{a120\_2mod[}\StringTok{"adapter\_length"}\NormalTok{] }\OtherTok{\textless{}{-}}\NormalTok{ a120\_2mod}\SpecialCharTok{$}\NormalTok{start }\SpecialCharTok{{-}} \DecValTok{1}
\NormalTok{a120\_4mod[}\StringTok{"adapter\_length"}\NormalTok{] }\OtherTok{\textless{}{-}}\NormalTok{ a120\_4mod}\SpecialCharTok{$}\NormalTok{start }\SpecialCharTok{{-}} \DecValTok{1}
\end{Highlighting}
\end{Shaded}

\subsubsection{Histograms of all the data in each
set}\label{histograms-of-all-the-data-in-each-set}

\begin{Shaded}
\begin{Highlighting}[]
\NormalTok{datasets }\OtherTok{\textless{}{-}} \FunctionTok{list}\NormalTok{(}
  \AttributeTok{a60\_30 =}\NormalTok{ a60\_30,}
  \AttributeTok{a60\_60 =}\NormalTok{ a60\_60,}
  \AttributeTok{a60\_unmod =}\NormalTok{ a60\_unmod,}
  \AttributeTok{a120\_1mod =}\NormalTok{ a120\_1mod,}
  \AttributeTok{a120\_2mod =}\NormalTok{ a120\_2mod,}
  \AttributeTok{a120\_4mod =}\NormalTok{ a120\_4mod,}
  \AttributeTok{a120\_unmod =}\NormalTok{ a120\_unmod}
\NormalTok{)}

\ControlFlowTok{for}\NormalTok{ (name }\ControlFlowTok{in} \FunctionTok{names}\NormalTok{(datasets)) \{}
\NormalTok{  adapter\_length }\OtherTok{\textless{}{-}}\NormalTok{ datasets[[name]]}\SpecialCharTok{$}\NormalTok{adapter\_length}
  
  \FunctionTok{hist}\NormalTok{(adapter\_length, }
       \AttributeTok{probability =} \ConstantTok{TRUE}\NormalTok{, }
       \AttributeTok{main =} \FunctionTok{paste}\NormalTok{(}\StringTok{"Density Plot of adapter lengths {-}"}\NormalTok{, name), }
       \AttributeTok{xlab =} \StringTok{"Adapter length"}\NormalTok{, }
       \AttributeTok{ylab =} \StringTok{"Density"}\NormalTok{, }
       \AttributeTok{xlim =} \FunctionTok{c}\NormalTok{(}\DecValTok{0}\NormalTok{, }\DecValTok{8000}\NormalTok{),}
       \AttributeTok{ylim =} \FunctionTok{c}\NormalTok{(}\DecValTok{0}\NormalTok{, }\FloatTok{0.0013}\NormalTok{),}
       \AttributeTok{col =} \StringTok{"lightblue"}\NormalTok{,}
       \AttributeTok{border =} \StringTok{"black"}\NormalTok{,         }
       \AttributeTok{breaks =} \DecValTok{200}\NormalTok{)              }
  
  \FunctionTok{lines}\NormalTok{(}\FunctionTok{density}\NormalTok{(adapter\_length), }\AttributeTok{col =} \StringTok{"red"}\NormalTok{, }\AttributeTok{lwd =} \DecValTok{2}\NormalTok{)}
\NormalTok{\}}
\end{Highlighting}
\end{Shaded}

\includegraphics{Adapter-lengths-for-all-datasets_files/figure-latex/unnamed-chunk-3-1.pdf}
\includegraphics{Adapter-lengths-for-all-datasets_files/figure-latex/unnamed-chunk-3-2.pdf}
\includegraphics{Adapter-lengths-for-all-datasets_files/figure-latex/unnamed-chunk-3-3.pdf}
\includegraphics{Adapter-lengths-for-all-datasets_files/figure-latex/unnamed-chunk-3-4.pdf}
\includegraphics{Adapter-lengths-for-all-datasets_files/figure-latex/unnamed-chunk-3-5.pdf}
\includegraphics{Adapter-lengths-for-all-datasets_files/figure-latex/unnamed-chunk-3-6.pdf}
\includegraphics{Adapter-lengths-for-all-datasets_files/figure-latex/unnamed-chunk-3-7.pdf}

\subsubsection{Histograms of all the data with log
transformation}\label{histograms-of-all-the-data-with-log-transformation}

\begin{Shaded}
\begin{Highlighting}[]
\ControlFlowTok{for}\NormalTok{ (name }\ControlFlowTok{in} \FunctionTok{names}\NormalTok{(datasets)) \{}
\NormalTok{  adapter\_length }\OtherTok{\textless{}{-}}\NormalTok{ datasets[[name]]}\SpecialCharTok{$}\NormalTok{adapter\_length}
  
  \CommentTok{\# Remove values ≤ 0}
\NormalTok{  adapter\_length }\OtherTok{\textless{}{-}}\NormalTok{ adapter\_length[adapter\_length }\SpecialCharTok{\textgreater{}} \DecValTok{0}\NormalTok{]}
  
\NormalTok{  log\_adapter\_length }\OtherTok{\textless{}{-}} \FunctionTok{log10}\NormalTok{(adapter\_length) }
  
  \FunctionTok{hist}\NormalTok{(log\_adapter\_length, }
       \AttributeTok{probability =} \ConstantTok{TRUE}\NormalTok{, }
       \AttributeTok{main =} \FunctionTok{paste}\NormalTok{(}\StringTok{"Density Plot of log{-}transformed adapter lengths {-}"}\NormalTok{, name), }
       \AttributeTok{xlab =} \StringTok{"Log10(Adapter Length)"}\NormalTok{,}
       \AttributeTok{ylab =} \StringTok{"Density"}\NormalTok{, }
       \AttributeTok{border =} \StringTok{"black"}\NormalTok{, }
       \AttributeTok{col =} \StringTok{"lightblue"}\NormalTok{,}
       \AttributeTok{xlim =} \FunctionTok{c}\NormalTok{(}\DecValTok{3}\NormalTok{, }\DecValTok{4}\NormalTok{),}
       \AttributeTok{ylim =} \FunctionTok{c}\NormalTok{(}\DecValTok{0}\NormalTok{, }\DecValTok{7}\NormalTok{),}
       \AttributeTok{breaks =} \DecValTok{100}\NormalTok{)                  }
  
  \FunctionTok{lines}\NormalTok{(}\FunctionTok{density}\NormalTok{(log\_adapter\_length), }\AttributeTok{col =} \StringTok{"red"}\NormalTok{, }\AttributeTok{lwd =} \DecValTok{2}\NormalTok{)}
\NormalTok{\}}
\end{Highlighting}
\end{Shaded}

\includegraphics{Adapter-lengths-for-all-datasets_files/figure-latex/unnamed-chunk-4-1.pdf}
\includegraphics{Adapter-lengths-for-all-datasets_files/figure-latex/unnamed-chunk-4-2.pdf}
\includegraphics{Adapter-lengths-for-all-datasets_files/figure-latex/unnamed-chunk-4-3.pdf}
\includegraphics{Adapter-lengths-for-all-datasets_files/figure-latex/unnamed-chunk-4-4.pdf}
\includegraphics{Adapter-lengths-for-all-datasets_files/figure-latex/unnamed-chunk-4-5.pdf}
\includegraphics{Adapter-lengths-for-all-datasets_files/figure-latex/unnamed-chunk-4-6.pdf}
\includegraphics{Adapter-lengths-for-all-datasets_files/figure-latex/unnamed-chunk-4-7.pdf}

\end{document}
